%%%%%%%%%%%%%%%%%%%%%%%%%%%%%%%%%%%%%%%%%%%%%%%%%
% openshift.tex
% 
% Latex source to be converted to HTML by tex4ht
%
%%%%%%%%%%%%%%%%%%%%%%%%%%%%%%%%%%%%%%%%%%%%%%%%%
\documentclass{article}

%%%%%%%%%%%%%%%%%%%%%%%%%%%%%%%%%%%%%%%%%%%%%%%%%
% PREAMBLE
%%%%%%%%%%%%%%%%%%%%%%%%%%%%%%%%%%%%%%%%%%%%%%%%%
% Import common configurations and custom functions
%%%%%%%%%%%%%%%%%%%%%%%%%%%%%%%%%%%%%%%%%%%%%%%%%
% PREAMBLE
%%%%%%%%%%%%%%%%%%%%%%%%%%%%%%%%%%%%%%%%%%%%%%%%%
\usepackage{hyperref}
\usepackage{listings}
\usepackage{color}
\usepackage[dvipsnames]{xcolor}
\usepackage{float}
\usepackage{graphicx}
\usepackage{tikz,tikz-dependency}
\usepackage{ifthen}
\usetikzlibrary{calc}
\usetikzlibrary{automata,positioning,external,shapes,arrows,chains,matrix,scopes,backgrounds}

% The following is for tikz externalization configuration which allows
% tikZ diagrams to render correctly while being processed by htlatex 
% http://tex.stackexchange.com/questions/158551/using-htlatex-with-tikz-dependency#158921
%%%%%%%%%%%%%%%%%%%%%%%%%%%%%%%%%%%%%%%%%%%%%%%%%
\tikzset{
    tex4ht inc/.style={
        /pgf/images/include external/.code={%
            \includegraphics[]{##1.svg}%
        }

    }
}
\tikzset{
 external/system call/.add={}
      ; inkscape -z -f "\image.pdf" -l "\image.svg"
}
\makeatletter
\@ifpackageloaded{tex4ht}{
    \tikzexternalize[mode=only graphics]
}{
    \tikzexternalize
}
\makeatother

% Removes section numbers
\makeatletter
\renewcommand\thesection{}
%\renewcommand\thesubsection{\@arabic\c@section.\@arabic\c@subsection}
\renewcommand\thesubsection{}
\makeatother

% lstlisting configs for code syntax formatting
\lstdefinestyle{Java} {
  frame=tb,
  language=Java,
  aboveskip=3mm,
  belowskip=3mm,
  showstringspaces=false,
  columns=flexible,
  basicstyle={\large\ttfamily},
  numbers=none,
  numberstyle=\textcolor{gray},
  keywordstyle=\textcolor{red!75},
  commentstyle=\textcolor{dkgreen},
  stringstyle=\textcolor{blue},
  moredelim=[is][\textcolor{black!75}]{|}{|},
  breaklines=false,
  breakatwhitespace=true,
  tabsize=3
}
%%%%%%%%%%%%%%%%%%%%%%%%%%%%%%%%%%%%%%%%%%%%%%%%%
% COLOR DEFINITIONS
%%%%%%%%%%%%%%%%%%%%%%%%%%%%%%%%%%%%%%%%%%%%%%%%%
\tikzstyle{red}     = [fill=red!30!white]
\tikzstyle{orange}  = [fill=orange!30!white]
\tikzstyle{yellow}  = [fill=yellow!30!white]
\tikzstyle{green}   = [fill=green!30!white]
\tikzstyle{blue}    = [fill=blue!30!white]
\tikzstyle{teal}    = [fill=teal!30!white]
\tikzstyle{purple}  = [fill=purple!30!white]
\tikzstyle{magenta} = [fill=magenta!30!white]
\tikzstyle{gray}    = [fill=gray!30!white]
\tikzstyle{black}   = [fill=black!30!white]

%%%%%%%%%%%%%%%%%%%%%%%%%%%%%%%%%%%%%%%%%%%%%%%%%
% DEFINE STACK COLORS
%%%%%%%%%%%%%%%%%%%%%%%%%%%%%%%%%%%%%%%%%%%%%%%%%
\definecolor{DEFINE_USR_COLOR}{HTML}{006699}
\definecolor{DEFINE_KRN_COLOR}{HTML}{003366}
\definecolor{DEFINE_HW_COLOR}{HTML}{000033}
\definecolor{DEFINE_CET_COLOR}{HTML}{339999}

\tikzstyle{DK_COLOR}  = [fill=blue!30!white]
\tikzstyle{ARQ_COLOR} = [fill=gray!30!white]
\tikzstyle{OS_COLOR}  = [fill=black!50!white]
\tikzstyle{USR_COLOR} = [fill=DEFINE_USR_COLOR]
\tikzstyle{KRN_COLOR} = [fill=DEFINE_KRN_COLOR]
\tikzstyle{HW_COLOR}  = [fill=DEFINE_HW_COLOR]
\tikzstyle{CET_COLOR} = [fill=DEFINE_CET_COLOR]
\tikzstyle{HM_COLOR}  = [fill=cyan!30!white]
%%%%%%%%%%%%%%%%%%%%%%%%%%%%%%%%%%%%%%%%%%%%%%%%%
% STACK DIAGRAM DEFINITIONS
%%%%%%%%%%%%%%%%%%%%%%%%%%%%%%%%%%%%%%%%%%%%%%%%%
\tikzstyle{layer}      = [rectangle, thick, rounded corners]
\tikzstyle{core_stack} = [layer, minimum width=11cm,minimum height=1cm, text = white]
\tikzstyle{user_stack} = [layer, minimum width=4cm,minimum height=1cm]
\tikzstyle{app_stack}  = [layer, minimum width=4.5cm,minimum height=3cm]
\tikzstyle{container}  = [layer, purple,minimum width=0.5cm,minimum height=0.5cm]

\tikzstyle{OS_LAYER}  = [app_stack, mw, OS_COLOR, label={[label distance=-0.75cm]270:OpenShift}]
\tikzstyle{ARQ_LAYER} = [app_stack, ARQ_COLOR,label={[label distance=-0.75cm]270:Arquillian}]
\tikzstyle{CET_LAYER} = [user_stack, CET_COLOR]
\tikzstyle{USR_LAYER} = [core_stack, USR_COLOR, minimum height=4.5cm, label={[label distance=-0.75cm]270:\color{white}Userspace}]
\tikzstyle{KRN_LAYER} = [core_stack, KRN_COLOR, below= 0cm of USR]
\tikzstyle{HW_LAYER}  = [core_stack, HW_COLOR, below of=KRN]
\tikzstyle{DK_LAYER}  = [user_stack, DK_COLOR]
\tikzstyle{HM_LAYER}  = [user_stack, HM_COLOR]

%%% NEW DEFS
\tikzstyle{OS_LAYER2} = [app_stack, OS_COLOR,fill=black!50!white, label={[label distance=-0.75cm]270:OpenShift}]


\tikzstyle{c_temp} = [layer,minimum width=0.5cm,minimum height=0.5cm]
\tikzstyle{c_empt} = [c_temp]
\tikzstyle{c_move} = [c_temp, draw, dotted]
\tikzstyle{c_full} = [c_temp, purple, draw=black, thin]
\tikzstyle{mw}     = [minimum width=5.0cm]

%%%%%%%%%%%%%%%%%%%%%%%%%%%%%%%%%%%%%%%%%%%%%%%%
% Tikz Functions
%%%%%%%%%%%%%%%%%%%%%%%%%%%%%%%%%%%%%%%%%%%%%%%%
\newcommand{\moby}[4] {
  \xdef\cmax{7}
  \def\id{#1}                   % #1 = id 
  \pgfmathsetmacro\nc{int(#2)}  % #2 = number of containers
  \pgfmathsetmacro\nfc{int(#3)} % #3 = number of full containers 
                                % #4 = location
  % Initialize container nodes
  \foreach \x in {0,...,7}{
    \node[] (C\x_\id) {};
  }
  % Instantiate moby and container nodes
  \node[DK_LAYER, DK_COLOR, #4] (DK_\id) {Moby};
  \node[above left= 0.1cm and 0cm of DK_\id.north west] (sp) {};

  \xdef\sp{sp};     % starting position
  \xdef\ct{c_full}; % container type
  \foreach \x in {0,...,\cmax} {
    {\ifthenelse{\x < \nc}
      { {\ifthenelse{\x < \nfc }
          {\xdef\ct{c_full};}
          {\xdef\ct{c_move};}
        }
      }
      { \xdef\ct{c_empt};}
    }
    \node[\ct, right=of \sp] (C\x_\id) {};
    \xdef\sp{C\x_\id};
  };

  {\ifthenelse{\nc < 4}{}{
  %Container label with spanning arrows 
  \node[above= 0.5cm of DK_\id.north ] (C_LABEL_\id) {Containers};
  \path[->, to path={-| (\tikztotarget)}]
  (C_LABEL_\id.west) edge (C0_\id.north)
  (C_LABEL_\id.east) edge (C7_\id.north);
  }};
}
\newcommand{\host}[4]{
  \def\id{#1}  % #1 = id
  \def\nc{#2}  % #2 = number of containers
  \def\nfc{#3} % #3 = number of full containers
               % #4 = location
  % Initialize nodes
  \foreach \x in {0,...,7}
    \node[] (C\x_\id) {};
    \node[] (DK_\id) {};
    \node[] (OS_\id) {};

  \node[OS_LAYER,fit=(DK_\id)] at (#4) (OS_\id) {};
  \moby{#1}{#2}{#3}{above = 0cm of OS_\id.center, anchor=center};
  \node[HM_LAYER, mw, below= 0cm of OS_\id.south, anchor=north] (HM_\id) {Host Machine \id};
}

\newcommand{\OpenShift}[4]{
  \def\id{#1}  % #1 = id
  \def\nc{#2}  % #2 = number of containers
  \def\nfc{#3} % #3 = number of full containers
               % #4 = location
  % Initialize nodes
  \foreach \x in {0,...,7}
    \node[] (C\x_\id) {};
    \node[] (DK_\id) {};
    \node[] (OS_\id) {};

  \node[OS_LAYER,fit=(DK_\id), #4] (OS_\id) {};
  \moby{#1}{#2}{#3}{above = 0cm of OS_\id.center, anchor=center};
}

\newcommand{\Arquillian}[2]{
  \def\id{#1}  % #1 = id
               % #2 = location

  \node[] (CET_\id) {};
  \node[ARQ_LAYER, fit=(CET_\id), #2] (ARQ_\id) {};
  \node[CET_LAYER, above = 0cm of ARQ_\id.center, anchor=center] (CET_\id) {CE-Testsuite};
}

\newcommand{\pods}[5]{
  \xdef\id{#1}
  \pgfmathtruncatemacro{\numPods}{#2-1}
  \xdef\numCons{#3-1};
  \pgfmathtruncatemacro{\cpp}{2 - 1} %containers per pod
  \xdef\cfit{} % containers for pod to fit
  \xdef\pfit{} % pods for namespace to fit

  \node[#4] (BASE) {};
  \xdef\lcp{BASE}; % last container position
  \xdef\sbc{}; % space between container
  \pgfmathsetmacro\r{#3}
 % Initialize pods
  \foreach \x in {0,...,\numPods}{
    \pgfmathsetmacro\pods{int(#2 - \x))}
    {\ifthenelse{\r > \pods }
    { % if 
      \xdef\numCons{1}
      \pgfmathsetmacro\r{int(\r - 2)}
      \xdef\r{\r}
    }
    { % else
      \xdef\numCons{0};
      \pgfmathsetmacro\r{int(\r - 1)}
      \xdef\r{\r}
    }
    };
    %\node[] at (-5,\x) {\r \pods \numCons};

    \foreach \y in {0,...,\numCons}{
      \node[c_full, right= \sbc of \lcp] (P\x_C\y_#1) {};
      \xdef\lcp{P\x_C\y_#1}
      \xdef\cfit{\cfit(P\x_C\y_#1)} % add container to be fitted
      \xdef\sbc{}; %space between containers
    }

    % randomly generate color for pod
    \pgfmathparse{rnd}
    \pgfmathtruncatemacro{\cc}{(\id)*0.4}
    \xdefinecolor{rColor}{rgb}{\cc, 0.7, \pgfmathresult}

    \node[pod, fill=rColor, fit=\cfit] (P\x_\id) {};
    %\pgfmathparse{rnd * \x}

    \xdef\cfit{};
    \xdef\sbc{0cm and 0.5cm}; %space between containers
    \xdef\pfit{\pfit(P\x_\id)} % add container to be fitted
    }
    #5
}

\newcommand{\namespace}[4]{
  \pods{#1}{#2}{#3}{#4}{\node[fit=\pfit] (NS_#1) {};};
  \node[NS, fit=\pfit] (NS_#1) {};
  \node[above=0cm of NS_#1.north] (NS_#1_LABEL) {Namespace #1};
  %\node[NS, fill=gray!10, fit=\pfit] (NS_#1) {};
  \pods{#1}{#2}{#3}{#4}{};
  %\pods{#1}{#2}{#3}{#4}{};
}


%%%%%%%%%%%%%%%%%%%%%%%%%%%%%%%%%%%%%%%%%%%%%%%%%
% DOCUMENT
%%%%%%%%%%%%%%%%%%%%%%%%%%%%%%%%%%%%%%%%%%%%%%%%%
\begin{document}

\centerline{\sc \large OpenShift}
\centerline{\sc Architecture for the Layperson }
\centerline{\url{https://github.com/openshift/origin}}

\vspace{1pc}

\section{Overview}
\hspace{3pc} Openshift is a container platform which facilitates the development and hosting of software applications across
distributed systems. Whether these distributed systems be physical hosts, virtual machines or any combination of 
both, OpenShift reaps the benefits of available computing resources while abstracting the unattractive details 
of managing it. 

\begin{figure}
\begin{tikzpicture}[node distance = 1cm and 0cm]
  \tikzstyle{HYP_COLOR} = [fill=gray!50!black]
  
  % OpenShift layer
  \openshift{OS0}{3.2}{}{}
  %----------------------------------------------%
  % HOST 2
  \openshift{OS1}{3.8}{above left = -1.5cm and 2.5cm of OS0}{}
  \node[USR_COLOR, layer,minimum height = 2.5cm,
   label={[label distance=-0.55cm]270:\color{white}Userspace}, above = 0.45 cm  of LOCATION ] (OS1_USR) {};
  \tikzstyle{mw} = [minimum width= 3cm]
  \tikzstyle{mh} = [minimum width=0.25cm]
  \node[box, HYP_COLOR, mw, above=0.5cm of OS1_USR.south] (OS1_HYPER) {\footnotesize Hypervisor};
  \node[box, KRN_COLOR, mw, above= 0cm and 0cm of OS1_HYPER] (OS1_KRN) {\footnotesize Kernel};
  \node[box,USR_COLOR, draw=white, mw, minimum height = 1cm, above= 0cm and 0cm of OS1_KRN,
    label={[label distance=-0.55cm]270:\color{white}\footnotesize Userspace}] (OS1_USR2) {};
  \node[OS_LAYER, minimum width = 1.5 , above = 0.5cm and 0.0cm of OS1_USR2.south] (OS1_label2) {\scriptsize OpenShift};

  %----------------------------------------------%
  %\openshift{OS2}{3.2}{right = 0cm and 2cm of OS0}{}
 
  %----------------------------------------------%
  % HOST 3
  \openshift{OS2}{3.8}{ above right = -1.5cm and 2.5cm of OS0}{}
  \node[USR_COLOR, layer,minimum height = 2.5cm,
   label={[label distance=-0.55cm]270:\color{white}Userspace}, above = 0.45 cm  of LOCATION ] (OS2_USR) {};
  
  \tikzstyle{mw} = [minimum width=1.85cm]
  \tikzstyle{mh} = [minimum width=0.25cm]
  
  \node[OS_LAYER, mw, above left = 0.5cm and 0.0cm of OS2_USR.south] (\id_label) {\footnotesize OpenShift};
  \node[box, HYP_COLOR, mw, right= 0cm and 0cm of OS2_label] (HYPER) {\footnotesize Hypervisor};
  \node[box, KRN_COLOR, mw, above= 0cm and 0cm of HYPER] (OS2_KRN) {\footnotesize Kernel};
  \node[box,USR_COLOR, draw=white, mw, minimum height = 1cm, above= 0cm and 0cm of OS2_KRN,
    label={[label distance=-0.55cm]270:\color{white}\footnotesize Userspace}] (OS2_USR2) {};
  \node[OS_LAYER, minimum width = 0.5 , above = 0.5cm and 0.0cm of OS2_USR2.south] (OS2_label2) {\scriptsize OpenShift};
  %-----------------------------------------------%
  %\node[blue, rectangle, right= of OS0] (TEST) {};

  %\node[] at (0,0) {0,0};

  \openshift{OS3}{3.2}{right = 7cm of OS0}{}   

  \project{PRJ1}{0.5}{8}{4}{above left= 2.75cm and 1.50cm of OS0}
  \project{PRJ1}{0.5}{8}{4}{above left= 2.75cm and 1.50cm of OS0}
  \project{PRJ2}{0.5}{4}{2}{above left= 2.75cm and 1.50cm of OS3}
  \project{PRJ2}{0.5}{4}{2}{above left= 2.75cm and 1.50cm of OS3}
  %\project{PRJ2}{0.5}{6}{3}{above right= 2.5cm and 0.0cm of OS0}

  % Connector nodes
  \node[above = 1.45cm of OS2_label2] (PRJ1P3_CONN) {};
  \node[below = 0.75cm of PRJ1P1]     (PRJ1P1_CONN) {};
  \node[left  = 1.00cm of PRJ1P0]     (PRJ1P0_CONN) {};
  \node[above = 0.50cm of PRJ1P2]     (PRJ1P2_CONN) {};
  \node[above right = 4.50cm and 0.5cm of OS3.north]        (PRJ1P2_CONN2) {};
  \node[above = 3.15cm of OS3.north]  (PRJ2P1_CONN) {};
  \node[above = 2.00cm of OS0]        (PRJ2P0_CONN) {};
  \node[below = 0.25cm of PRJ2P0]     (PRJ2P0_CONN2) {};

  \tikzstyle{link} = [-, thick, opacity=0.3]

  \path[link, to path={|- (\tikztotarget)}]
    (PRJ1P3.east) edge (PRJ1P3_CONN.center)
    (PRJ1P3_CONN.center) edge (OS2_label2.north)
    (PRJ1P1.south) edge (PRJ1P1_CONN.center)
    (PRJ1P1_CONN.center) edge (OS1_label2.east)
    (PRJ1P0.west) edge (PRJ1P0_CONN.center)
    (PRJ1P0_CONN.center) edge (OS1_label2.north)
    (PRJ1P2.north) edge (PRJ1P2_CONN.center)
    (PRJ1P2_CONN.center) edge (PRJ1P2_CONN2.center)
    (PRJ1P2_CONN2.center) edge (OS3_label.north east)
    (OS0_label.north) edge (PRJ2P0_CONN.center)
    (PRJ2P0_CONN.center) edge (PRJ2P0_CONN2.center)
    (PRJ2P0.south) edge (PRJ2P0_CONN2.center)
    (OS3_label.north) edge (PRJ2P1_CONN.center)
    (PRJ2P1.east) edge (PRJ2P1_CONN.center);    

  \node[below = 0.50cm of PRJ2P1] (POD_LABEL)  {Pod};
  \path[->] (POD_LABEL) edge (PRJ2P1);

  \node[above = 0.00cm of PRJ1] (PRJ_LABEL)  {Project};
  %\path[->] (PRJ_LABEL) edge (PRJ1);

\end{tikzpicture}
\end{figure}

%Development within the OpenShift environment begins and ends with virtual computer hosts known as pods. This is where
%programmers interact with and host code. The computing resources backing these pods is determined by the OpenShift cluster.
%Pods are indivisible and must reside on one machine; however, this is not the case with groups of pods known as a project.

\section{Core Components}
Within the OpenShift environment, programmers isolate their programs into virtual instances known
as pods. Pods emulate the behavoir of a programmer's physical or virtual host by leveraging aspects 
of the host's kernel. A pod consists of a group of processes that can share disk space. A pod is 
assigned its own ip address. 
%The whole purpose of container platforms is to relief programmers of mundane task of environmental upkeep. 
%\iffalse
%Distributed openshift and pod diagram
%%%%%%%%%%%%%%%%%%%%%%%%%%%%%%%%%%%%%%%%%%%%%%%%%
% Core Components
%%%%%%%%%%%%%%%%%%%%%%%%%%%%%%%%%%%%%%%%%%%%%%%%%
\begin{figure}
\begin{tikzpicture}[node distance = 1cm and 0cm]
  % OpenShift layer
  \tikzstyle{OS_LAYER} = [app_stack, OS_COLOR, minimum width = 14cm, minimum height = 6.0cm]
  \tikzstyle{image} = [c_full, fill=white!20, opacity=0.5]
  \tikzstyle{mw} = [minimum width=15.0cm]
  \tikzstyle{mh} = [minimum height=7.5cm]
  \tikzstyle{link} = [-, thick, opacity=0.3]

  \node[USR_LAYER, mw, mh] (USR) {};
  \node[KRN_LAYER, mw, below = 0cm and 0cm of USR.south] (KRN) {Kernel};
  \node[HW_LAYER,  mw, minimum height = 2cm, below = 0cm and 0cm of KRN.north, anchor=north] (HW) {Hardware};

  \node[OS_LAYER, above=1cm of USR.south] (OS) {\Large \color{white} OpenShift};
  \node [below = 0cm of HW.south] (HM) {Physical Machine};
  \path[->, thin, to path={-| (\tikztotarget)}]
    (HM.west) edge (HW.south west)
    (HM.east) edge (HW.south east);

  \node[box, minimum height = 1.75cm, minimum width = 5.5cm, DK_COLOR, below left = 1.0cm and 1.15cm of OS.center,
       label={[label distance=-0.65cm]270:Container Engine}] (DK) {};
  \containers{0}{8}{8}{above left= 0.75cm and 2cm of DK.south};
  
  \project{PRJ0}{0.5}{8}{4}{above left = 2.00cm and 0.0cm of DK}
  \project{PRJ0}{0.5}{8}{4}{above left = 2.00cm and 0.0cm of DK}
  \node[below = 0.0cm of PRJ0] (PRJ0_LABEL) {Project};

  \node[above right= -0.25cm and 0.40cm of PRJ0P3] (POD_LABEL) {Pod};
  \path[link, ->] (POD_LABEL) edge (PRJ0P3);

  \xdef\dis{0.75}
  \node[below = 2 cm of PRJ0P0C00] (CONN0_0) {};
  \node[below = 1.75 cm of PRJ0P0C01] (CONN1_0) {};
  \node[below = 1.5 cm of PRJ0P1C00] (CONN2_0) {};
  \node[below = 2 cm of PRJ0P1C01] (CONN3_0) {};
  \node[below = 2 cm of PRJ0P2C00] (CONN4_0) {};
  \node[below = 1.75 cm of PRJ0P2C01] (CONN5_0) {};
  \node[below = 1.75 cm of PRJ0P3C00] (CONN6_0) {};
  \node[below = 2 cm of PRJ0P3C01] (CONN7_0) {};

  \path[link, to path={|- (\tikztotarget)}]
    (C0_0.north) edge (CONN0_0.center)
    (C1_0.north) edge (CONN1_0.center)
    (C2_0.north) edge (CONN2_0.center)
    (C3_0.north) edge (CONN3_0.center)
    (C4_0.north) edge (CONN4_0.center)
    (C5_0.north) edge (CONN5_0.center)
    (C6_0.north) edge (CONN6_0.center)
    (C7_0.north) edge (CONN7_0.center);

  \path[link, ->]
    (CONN0_0.center) edge (PRJ0P0C00.south)
    (CONN1_0.center) edge (PRJ0P0C01.south)
    (CONN2_0.center) edge (PRJ0P1C00.south)
    (CONN3_0.center) edge (PRJ0P1C01.south)
    (CONN4_0.center) edge (PRJ0P2C00.south)
    (CONN5_0.center) edge (PRJ0P2C01.south)
    (CONN6_0.center) edge (PRJ0P3C00.south)
    (CONN7_0.center) edge (PRJ0P3C01.south);


  %\path[link, ->] (PRJ0P3.east) edge (P0);
  \node[right = 0.0cm and 3.55cm of PRJ0P3.10] (CONN0_P3) {};
  \node[right = 0.0cm and 3.25cm of PRJ0P3.350](CONN1_P3) {};

  \tikzstyle{d_seg_color} = [fill=black!10]
  \tikzstyle{d_seg} = [cylinder, d_seg_color,draw, shape border rotate=90, thin, minimum width = 1.7cm, minimum height = 1.5cm]

  \node[d_seg, right= 0cm and 2cm of HW.center] (D0) {};
  \node[below= 0cm of D0.center ] (D0_label) {Disk};

  \tikzstyle{link} = [-, thick, white, opacity=0.9]
  \path[link]  (PRJ0P3.10) edge (CONN0_P3.center);
  \path[link, ->]  (CONN0_P3.center) edge (D0.80);
  \path[link, <-]  (PRJ0P3.350) edge (CONN1_P3.center);
  \path[link]  (CONN1_P3.center) edge (D0.100);

  \tikzstyle{port} = [rectangle, minimum height = 0.05cm, minimum width = 4cm, fill=white]
  %\node[port, right= 0.0cm and 0cm of PRJ0P3.east]  (PORT0) {\tiny 101010101010101011010101010101010};
  \node[port, rotate=90, minimum width = 1cm, above= 0.0cm and 1cm of PRJ0P0.north, anchor=west] (PORT0) {\tiny 010101010};
  \node[port, rotate=90, minimum width = 1cm, above= 0.0cm and 1cm of PRJ0P1.north, anchor=west] (PORT1) {\tiny 010101010};
  \node[port, rotate=90, minimum width = 1cm, above= 0.0cm and 1cm of PRJ0P2.north, anchor=west] (PORT2) {\tiny 010101010};
  \node[port, rotate=90, minimum width = 1cm, above= 0.0cm and 1cm of PRJ0P3.north, anchor=west] (PORT3) {\tiny 010101010};
  %\node[above right = 0cm and -0.85cm of PORT0] (PORT0_LABEL) {\scriptsize Port};
  \node[rotate=90, above right = -0.7cm and 0.0cm of PORT0] (PORT0_LABEL) {\scriptsize Port};
  \node[rotate=90, above right = -0.7cm and 0.0cm of PORT1] (PORT1_LABEL) {\scriptsize Port};
  \node[rotate=90, above right = -0.7cm and 0.0cm of PORT2] (PORT1_LABEL) {\scriptsize Port};
  \node[rotate=90, above right = -0.7cm and 0.0cm of PORT3] (PORT3_LABEL) {\scriptsize Port};
\end{tikzpicture}
\caption{Core Components}
\end{figure}

%\section{Images}
%The most important component of container technology is what is known as an container image. If a container application is a cake 
%then the image is the recipe. One image can be instantiated multiple times make identical container applications just as a recipe 
%can be followed to make as many cakes as a programmer desires.

%\begin{figure}
%\begin{tikzpicture}
%  \cube{1}{2}{black}{};
%\end{tikzpicture}
%\end{figure}

%OpenShift keeps all of its recipes in a recipe cubby know as a registry. When a programmer wants a head start on their application,
%or write a new recipe, a image is chosen as the baseline

%- Add a section on the catalogue \n
%- Diff between OS image and docker image \n
%- How images are created \n
%- imagestreams \n
%- section on template \n
%- fix up cube code

%\section{TODO}
%\begin{enumerate}
%  \item Pods and Services
%  \item Builds and Image Streams
%  \item Container Registry
%  \item Web console
%  \item Builds and Image Streams
%  \item Deployments
%  \item Templates
%\end{enumerate}

\begin{figure}
\begin{tikzpicture}[node distance = 1cm and 0cm]
  % OpenShift layer
  \tikzstyle{OS_LAYER} = [app_stack, OS_COLOR, minimum width = 14cm, minimum height = 6.5cm]
  \tikzstyle{image} = [c_full, fill=white!20, opacity=0.5]
  \tikzstyle{mw} = [minimum width=15.0cm]
  \tikzstyle{mh} = [minimum height=8.0cm]

  \node[USR_LAYER, mw, mh] (USR) {};
  \node[KRN_LAYER, mw, below = 0cm and 0cm of USR.south] (KRN) {Kernel};
  \node[HW_LAYER,  mw, below = 0cm and 0cm of KRN.north, anchor=north] (HW) {Hardware};

  %\node[HM_LAYER, mw] (HM_1) {Host Machine 1};
  %\node[HM_LAYER, mw, right=of HM_1] (HM_2) {Host Machine 2};
  %\node[HM_LAYER, mw, right=of HM_2] (HM_3) {Host Machine 3};
  \node[OS_LAYER, above=1cm of USR.south] (OS) {\Large \color{white} OpenShift};
  %\node[HM_LAYER, mw] (MB_1) {Host Machine 1};

  \project{PRJ0}{0.5}{4}{2}{above left = 0.75cm and 6.5cm of OS.south}

  %        id size rows cols location
  \registry{R0}{0.5}{3}{4}{above left= 0.35cm and 2.5cm of OS.south} %location  
  \path[->, opacity=0.5, thick] (PRJ0P1) edge (R0C10_west);

  \router{ROUTE0}{1.75}{above= 1.0cm of PRJ0P0}
  \path[->, thick] (PRJ0P0) edge (ROUTE0);

  \cube{C1}{0.5}{black}{above= 1cm and 0.0 cm of R0C20_north }; % location
  \path[->, thick, opacity=0.5] (R0C20_north) edge (C1_south);

  \project{PRJ1}{0.5}{1}{1}{above left = 1cm and 0.5cm of  C1_north}
  \node[above= 0cm of PRJ1.north] (PRJ1_label) {\small Project};
  \path[->, thick, opacity=0.5] (C1_north) edge (PRJ1.south);

  %     id size color    location
  \cube{C0}{0.5}{black}{right= 0cm and 3.5 cm of R0C02 }; % location
  \node[below= 0.10cm of C0_south] (C0_label) {\small Image};
  \path[->, thick, opacity=0.5] (R0C03_east) edge (C0_west);
  
  \catalog{CAT0}{1.75}{7}{above right = 1.75cm and 5.0cm of OS.south};
  \template{T0}{1.75}{blue}{left = 2.5cm of CAT0T3};
  \node[below= 0cm of T0_south] (T0_label) {\small Template};
  \path[->, thick, opacity=0.3] (CAT0T3_west) edge (T0_east);

  \project{PRJ2}{0.5}{6}{3}{above left= 2cm and 2.60cm of  T0_north}
  \path[->, thick, opacity=0.3] (T0_center) edge (PRJ2.south);

 % \path[->, thick, opacity=0.3] (CO_north) edge (T0_label);


\end{tikzpicture}
\end{figure}

%
%
%\section{Build}
%A build is an method of creating container images that can run on OpenShift

%- Need visual of an image
%- Need visual for build

%BuildConfig ---> Like Blueprint schematic
%
%\begin{figure}
%\begin{tikzpicture}[node distance = 1cm and 0cm]
%   \node[circle, fill=blue] (ph) {PlaceHolder};
%
%\end{tikzpicture}
%\caption{Build Config}
%\end{figure}

%\section{Container Registry}
%OpenShift is able to store the blueprints for applications in what is know as a registry

%\section{Image Stream}
%"Comprises of any number of container images identified by tages / Virtual view of related images"
%A collection of image references, a way to mass import container images into OpenShift
%Keeps track of images within registry

%importing a tag or image into an image stream?

---------------------------------------------------------------------------------------------------- \\
 WIP \\
---------------------------------------------------------------------------------------------------- \\


\end{document}

