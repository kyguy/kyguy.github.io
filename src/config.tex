%%%%%%%%%%%%%%%%%%%%%%%%%%%%%%%%%%%%%%%%%%%%%%%%%
% PREAMBLE
%%%%%%%%%%%%%%%%%%%%%%%%%%%%%%%%%%%%%%%%%%%%%%%%%
\usepackage{hyperref}
\usepackage{listings}
\usepackage{color}
\usepackage[dvipsnames]{xcolor}
\usepackage{float}
\usepackage{graphicx}
\usepackage{tikz,tikz-dependency}
\usepackage{ifthen}
\usetikzlibrary{calc}
\usetikzlibrary{automata,positioning,external,shapes,arrows,chains,matrix,scopes,backgrounds}
\usepackage{enumerate}
% The following is for tikz externalization configuration which allows
% tikZ diagrams to render correctly while being processed by htlatex 
% http://tex.stackexchange.com/questions/158551/using-htlatex-with-tikz-dependency#158921
%%%%%%%%%%%%%%%%%%%%%%%%%%%%%%%%%%%%%%%%%%%%%%%%%
\tikzset{
    tex4ht inc/.style={
        /pgf/images/include external/.code={%
            \includegraphics[]{##1.svg}%
        }

    }
}
\tikzset{
 external/system call/.add={}
      ; inkscape -z -f "\image.pdf" -l "\image.svg"
}
\makeatletter
\@ifpackageloaded{tex4ht}{
    \tikzexternalize[mode=only graphics]
}{
    \tikzexternalize
}
\makeatother

% Removes section numbers
\makeatletter
\renewcommand\thesection{}
%\renewcommand\thesubsection{\@arabic\c@section.\@arabic\c@subsection}
\renewcommand\thesubsection{}
\makeatother

%%%%%%%%%%%%%%%%%%%%%%%%%%%%%%%%%%%%%%%%%%%%%%%%%
% COLOR DEFINITIONS
%%%%%%%%%%%%%%%%%%%%%%%%%%%%%%%%%%%%%%%%%%%%%%%%%
\tikzstyle{red}     = [fill=red!30!white]
\tikzstyle{orange}  = [fill=orange!30!white]
\tikzstyle{yellow}  = [fill=yellow!30!white]
\tikzstyle{green}   = [fill=green!30!white]
\tikzstyle{blue}    = [fill=blue!30!white]
\tikzstyle{teal}    = [fill=teal!30!white]
\tikzstyle{purple}  = [fill=purple!30!white]
\tikzstyle{magenta} = [fill=magenta!30!white]
\tikzstyle{gray}    = [fill=gray!30!white]
%\tikzstyle{black}   = [fill=black!30!white]

%%%%%%%%%%%%%%%%%%%%%%%%%%%%%%%%%%%%%%%%%%%%%%%%%
% DEFINE STACK COLORS
%%%%%%%%%%%%%%%%%%%%%%%%%%%%%%%%%%%%%%%%%%%%%%%%%
\definecolor{DEFINE_USR_COLOR}{HTML}{006699}
\definecolor{DEFINE_KRN_COLOR}{HTML}{003366}
\definecolor{DEFINE_HW_COLOR}{HTML}{000033}
\definecolor{DEFINE_CET_COLOR}{HTML}{339999}
\definecolor{DEFINE_CE_ARQ_COLOR}{HTML}{003333}

\tikzstyle{DK_COLOR}  = [fill=blue!30!white]
\tikzstyle{ARQ_COLOR} = [fill=gray!30!white]
\tikzstyle{OS_COLOR}  = [fill=black!50!white]
\tikzstyle{USR_COLOR} = [fill=DEFINE_USR_COLOR]
\tikzstyle{KRN_COLOR} = [fill=DEFINE_KRN_COLOR]
\tikzstyle{HW_COLOR}  = [fill=DEFINE_HW_COLOR]
\tikzstyle{CET_COLOR} = [fill=DEFINE_CET_COLOR]
\tikzstyle{CE_ARQ_COLOR} = [fill=DEFINE_CE_ARQ_COLOR]
\tikzstyle{HM_COLOR}  = [fill=cyan!30!white]
%%%%%%%%%%%%%%%%%%%%%%%%%%%%%%%%%%%%%%%%%%%%%%%%%
% STACK DIAGRAM DEFINITIONS
%%%%%%%%%%%%%%%%%%%%%%%%%%%%%%%%%%%%%%%%%%%%%%%%%
\tikzstyle{layer}      = [rectangle, thick, rounded corners]
\tikzstyle{core_stack} = [layer, minimum width=11cm,minimum height=1cm, text = white]
\tikzstyle{user_stack} = [layer, minimum width=4cm,minimum height=1cm]
\tikzstyle{app_stack}  = [layer, minimum width=4.5cm,minimum height=3cm]
\tikzstyle{container}  = [layer, purple,minimum width=0.5cm,minimum height=0.5cm]

\tikzstyle{OS_LAYER}  = [app_stack]
\tikzstyle{ARQ_LAYER} = [app_stacK]
\tikzstyle{CET_LAYER} = [user_stack, CET_COLOR]
\tikzstyle{USR_LAYER} = [core_stack, USR_COLOR, minimum height=4.5cm, label={[label distance=-0.75cm]270:\color{white}Userspace}]
\tikzstyle{KRN_LAYER} = [core_stack, KRN_COLOR, below= 0cm of USR]
\tikzstyle{HW_LAYER}  = [core_stack, HW_COLOR, below of=KRN]
\tikzstyle{DK_LAYER}  = [user_stack, DK_COLOR]
\tikzstyle{HM_LAYER}  = [user_stack, HM_COLOR]

%%% NEW DEFS
\tikzstyle{OS_LAYER2} = [app_stack, OS_COLOR,fill=black!50!white, label={[label distance=-0.75cm]270:OpenShift}]

\tikzstyle{c_temp} = [layer,minimum width=0.5cm,minimum height=0.5cm]
\tikzstyle{c_empt} = [c_temp]
\tikzstyle{c_move} = [c_temp, draw, dotted]
\tikzstyle{c_full} = [c_temp, purple, draw=black, thin]
\tikzstyle{mw}     = [minimum width=5.0cm]

\tikzstyle{c}    = [container, draw=black, thin]
\tikzstyle{pod}  = [thick, opacity=0.5, rounded corners]
\tikzstyle{NS}   = [thick, draw, dotted] %Namespace
\tikzstyle{blank}= [c_temp, OS_COLOR, thick, c_move]
\tikzstyle{pod}  = [thick, opacity=0.4, rounded corners, circle]
\tikzstyle{box} = [rectangle, thick, rounded corners, text= white]
\tikzstyle{link} = [-, thick, opacity=0.3]

\definecolor{keyword}{rgb}{0,0,0}
\definecolor{comment}{rgb}{0,0,0}
\definecolor{string}{rgb}{0,0,0}

% lstlisting configs for code syntax formatting
\lstdefinestyle{Java} {
  frame=tb,
  language=Java,
  aboveskip=3mm,
  belowskip=3mm,
  showstringspaces=false,
  columns=flexible,
  basicstyle={\large\ttfamily},
  numbers=none,
  numberstyle=\textcolor{gray},
  keywordstyle=\textcolor{red!75},
  commentstyle=\textcolor{dkgreen},
  stringstyle=\textcolor{blue},
  moredelim=[is][\textcolor{black!75}]{|}{|},
  breaklines=false,
  breakatwhitespace=true,
  tabsize=3,
  belowskip=-0.5em %\baselineskip
}

\lstdefinestyle{shell} {
  basicstyle=\ttfamily,
  showstringspaces=false,
  keywordstyle=\textcolor{keyword},
  commentstyle=\textcolor{comment},
  stringstyle=\textcolor{string},
  belowskip=-\baselineskip
  %belowskip=-\baselineskip
}

%%%%%%%%%%%%%%%%%%%%%%%%%%%%%%%%%%%%%%%%%%%%%%%%
% Tikz Functions
%%%%%%%%%%%%%%%%%%%%%%%%%%%%%%%%%%%%%%%%%%%%%%%%
\newcommand{\containers}[4] {
  \xdef\cmax{7}
  \def\id{#1}                   % #1 = id 
  \pgfmathsetmacro\nc{int(#2)}  % #2 = number of containers
  \pgfmathsetmacro\nfc{int(#3)} % #3 = number of full containers 
  \tikzstyle{LOCATION} = {#4}   % #4 = location
  \node[#4] (loc) {}; 

  \node[LOCATION] (loc) {};
  % Initialize container nodes
  \foreach \x in {0,...,7}{
    \node[] (C\x_\id) {};
  }

  \xdef\sp{loc};     % starting position
  \xdef\ct{c_full}; % container type
  \foreach \x in {0,...,\cmax} {
    {\ifthenelse{\x < \nc}
      { {\ifthenelse{\x < \nfc }
          {\xdef\ct{c_full};}
          {\xdef\ct{c_move};}
        }
      }
      { \xdef\ct{c_empt};}
    }
    \node[\ct, right=of \sp] (C\x_\id) {};
    \xdef\sp{C\x_\id};
  };

  %{\ifthenelse{\nc < 4}{}{
  %Container label with spanning arrows
  \node[fit=(C0_\id)(C1_\id)(C2_\id)(C3_\id)(C4_\id)(C5_\id)(C5_\id)(C6_\id)(C7_\id)] (all) {};
  \node[above= -0.15cm of all.north ] (C_LABEL_\id) {\small Containers};
  \path[->, to path={-| (\tikztotarget)}]
  (C_LABEL_\id.west) edge (C0_\id.north)
  (C_LABEL_\id.east) edge (C7_\id.north);
  
}

\newcommand{\moby}[4] {
  \xdef\cmax{7}
  \def\id{#1}                   % #1 = id 
  \pgfmathsetmacro\nc{int(#2)}  % #2 = number of containers
  \pgfmathsetmacro\nfc{int(#3)} % #3 = number of full containers 
                                % #4 = location
  % Initialize container nodes
  \foreach \x in {0,...,7}{
    \node[] (C\x_\id) {};
  }
  % Instantiate moby and container nodes
  \node[DK_LAYER, DK_COLOR, #4] (DK_\id) {Moby};
  \node[above left= 0.1cm and 0cm of DK_\id.north west] (sp) {};

  \xdef\sp{sp};     % starting position
  \xdef\ct{c_full}; % container type
  \foreach \x in {0,...,\cmax} {
    {\ifthenelse{\x < \nc}
      { {\ifthenelse{\x < \nfc }
          {\xdef\ct{c_full};}
          {\xdef\ct{c_move};}
        }
      }
      { \xdef\ct{c_empt};}
    }
    \node[\ct, right=of \sp] (C\x_\id) {};
    \xdef\sp{C\x_\id};
  };

  {\ifthenelse{\nc < 4}{}{
  %Container label with spanning arrows 
  \node[above= 0.5cm of DK_\id.north ] (C_LABEL_\id) {Containers};
  \path[->, to path={-| (\tikztotarget)}]
  (C_LABEL_\id.west) edge (C0_\id.north)
  (C_LABEL_\id.east) edge (C7_\id.north);
  }};
}
\newcommand{\host}[4]{
  \def\id{#1}  % #1 = id
  \def\nc{#2}  % #2 = number of containers
  \def\nfc{#3} % #3 = number of full containers
               % #4 = location
  % Initialize nodes
  \foreach \x in {0,...,7}
    \node[] (C\x_\id) {};
    \node[] (DK_\id) {};
    \node[] (OS_\id) {};

  \node[OS_LAYER,fit=(DK_\id)] at (#4) (OS_\id) {};
  \moby{#1}{#2}{#3}{above = 0cm of OS_\id.center, anchor=center};
  \node[HM_LAYER, mw, below= 0cm of OS_\id.south, anchor=north] (HM_\id) {Host Machine \id};
}

\newcommand{\OpenShift}[4]{
  \def\id{#1}  % #1 = id
  \def\nc{#2}  % #2 = number of containers
  \def\nfc{#3} % #3 = number of full containers
               % #4 = location
  % Initialize nodes
  \foreach \x in {0,...,7}
    \node[] (C\x_\id) {};
    \node[] (DK_\id) {};
    \node[] (OS_\id) {};

  \node[OS_LAYER,fit=(DK_\id), #4] (OS_\id) {};
  \moby{#1}{#2}{#3}{above = 0cm of OS_\id.center, anchor=center};
}

\newcommand{\Arquillian}[2]{
  \def\id{#1}  % #1 = id
               % #2 = location

  \node[] (CET_\id) {};
  \node[ARQ_LAYER, fit=(CET_\id), #2] (ARQ_\id) {};
  \node[CET_LAYER, above = 0cm of ARQ_\id.center, anchor=center] (CET_\id) {CE-Testsuite};
}

\newcommand{\pods}[5]{
  \xdef\id{#1}
  \pgfmathtruncatemacro{\numPods}{#2-1}
  \xdef\numCons{#3-1};
  \pgfmathtruncatemacro{\cpp}{2 - 1} %containers per pod
  \xdef\cfit{} % containers for pod to fit
  \xdef\pfit{} % pods for namespace to fit

  \node[#4] (BASE) {};
  \xdef\lcp{BASE}; % last container position
  \xdef\sbc{}; % space between container
  \pgfmathsetmacro\r{#3}
 % Initialize pods
  \foreach \x in {0,...,\numPods}{
    \pgfmathsetmacro\pods{int(#2 - \x)}
    {\ifthenelse{\r > \pods }
    { % if 
      \xdef\numCons{1}
      \pgfmathsetmacro\r{int(\r - 2)}
      \xdef\r{\r}
    }
    { % else
      \xdef\numCons{0};
      \pgfmathsetmacro\r{int(\r - 1)}
      \xdef\r{\r}
    }
    };
    %\node[] at (-5,\x) {\r \pods \numCons};

    \foreach \y in {0,...,\numCons}{
      \node[c_full, right= \sbc of \lcp] (P\x_C\y_#1) {};
      \xdef\lcp{P\x_C\y_#1}
      \xdef\cfit{\cfit(P\x_C\y_#1)} % add container to be fitted
      \xdef\sbc{}; %space between containers
    }

    % randomly generate color for pod
    \pgfmathparse{rnd}
    \pgfmathtruncatemacro{\cc}{(\id)*0.4}
    \xdefinecolor{rColor}{rgb}{\cc, 0.7, \pgfmathresult}

    \node[pod, fill=rColor, fit=\cfit] (P\x_\id) {};
    %\pgfmathparse{rnd * \x}

    \xdef\cfit{};
    \xdef\sbc{0cm and 0.5cm}; %space between containers
    \xdef\pfit{\pfit(P\x_\id)} % add container to be fitted
    }
    #5
}

\newcommand{\namespace}[4]{
  \pods{#1}{#2}{#3}{#4}{\node[fit=\pfit] (NS_#1) {};};
  \node[NS, fit=\pfit] (NS_#1) {};
  \node[above=0cm of NS_#1.north] (NS_#1_LABEL) {Namespace #1};
  %\node[NS, fill=gray!10, fit=\pfit] (NS_#1) {};
  \pods{#1}{#2}{#3}{#4}{};
  %\pods{#1}{#2}{#3}{#4}{};
}

\newcommand{\cube}[4] {
  \def\id{#1}
  \def\size{#2}
  \def\color{#3}
  \tikzstyle{location} = [#4]

  \def\x{\size * 0.4}
  \def\y{\size * 0.5}
  \def\z{\size * 1.0}

  \tikzstyle{coor} = [shape=coordinate]
  \node[circle, location] (\id) {};

  \node[coor, below left =\x cm and \y cm of \id] (S_\id) {S};
  \node[coor, above=\z cm and 0cm of S_\id] (A_\id) {A};
  \node[coor, above right=\y cm and \x cm of A_\id] (B_\id) {B};
  \node[coor, above right=\y cm and \x cm of S_\id] (C_\id) {C};
  \node[coor, right=0.0cm and \z cm of S_\id] (D_\id) {D};
  \node[coor, above=\z cm and 0.0cm of D_\id] (E_\id) {E};
  \node[coor, above right=\y cm and \x cm of E_\id] (F_\id) {F};
  \node[coor, above right=\y cm and \x cm of D_\id] (G_\id) {G};

  \tikzstyle{face} = [black]

  \draw[face, fill=\color!30] (S_\id) -- (C_\id) -- (G_\id) -- (D_\id) -- cycle; % Bottom Face
  \draw[face, fill=\color!30] (S_\id) -- (A_\id) -- (E_\id) -- (D_\id) -- cycle; % Back Face
  \draw[face, fill=\color!10] (S_\id) -- (A_\id) -- (B_\id) -- (C_\id) -- cycle; % Left Face
  \draw[face, fill=\color!20,opacity=0.8] (D_\id) -- (E_\id) -- (F_\id) -- (G_\id) -- cycle; % Right Face
  \draw[face, fill=\color!20,opacity=0.6] (C_\id) -- (B_\id) -- (F_\id) -- (G_\id) -- cycle; % Front Face
  \draw[face, fill=\color!20,opacity=0.8] (A_\id) -- (B_\id) -- (F_\id) -- (E_\id) -- cycle; % Top Face

  \node[coor] (\id_center) at ($(S_\id)!0.5!(F_\id)$)  {};
  \node[coor] (\id_north)  at ($(A_\id)!0.5!(F_\id)$)  {};
  \node[coor] (\id_east)   at ($(D_\id)!0.5!(F_\id)$)  {};
  \node[coor] (\id_south)  at ($(S_\id)!0.5!(G_\id)$)  {};
  \node[coor] (\id_west)   at ($(S_\id)!0.5!(B_\id)$)  {};

  %\node[fit=(S_\id)(F_\id)] (\id) {};
  %\tikzstyle{cube}= [fit=(S_\id)(A_\id)(B_\id)(C_\id)(D_\id)(E_\id)(F_\id)(G_\id)];
  %\node[circle, draw, location] (test\id) {};
}

% Registry \regID is labeled in the format R<Registry number> 
% (e.g. R0 would represent Registry 0) while the cubes are 
% separated into labels of R\idC\row\col 
% (e.g. R0C00 Cube row 0 and col 0 of Registry 0)
\newcommand{\registry}[5] {
  \def\regID{#1}
  \pgfmathsetmacro{\cubesize}{#2};
  \pgfmathsetmacro{\rows}{int(#3 - 1)}
  \pgfmathsetmacro{\cols}{int(#4 - 1)}
  \tikzstyle{loc} = [#5]

  \node[loc] (LOCATION) {};

  \foreach \r in {0,...,\rows}{
    \foreach \c in {0,...,\cols}{
      \pgfmathsetmacro{\vdis}{\r * \cubesize}
      \pgfmathsetmacro{\hdis}{\c * \cubesize}
      %\pgfmathsetmacro{\num}{int(\c + (\r * #4))}
      %\node[above right= \vdis cm and \hdis cm of LOCATION] (Cuve\num R) {};
      \cube{\regID C\r \c}{\cubesize}{gray}{above right= \vdis cm and \hdis cm of LOCATION};
    }
  }

  \node[fit=(S_\regID C00) (F_\regID C\rows \cols)] (\regID) {};
  \node[above= -0.10cm of \regID] (\regID_label) {\small Registry};
}

%\newcommand{\template}[4] {
%  \def\tempID{#1}
%  \pgfmathsetmacro{\size}{#2}
%  \pgfmathsetmacro{\minWidth}{\size * 0.75}
%  \def\color{#3}
%  \tikzstyle{tempID_location} = [#4]
%
%  \tikzstyle{template} = [rectangle, draw,fill = \color, minimum height = \size cm, 
%                          minimum width = \minWidth cm]
%  \node[template, tempID_location] (\tempID) {};
%}
%
% CAT0T00 = Catalog 0, template on row 0 column 0
%

\newcommand{\pod}[5]{
  \def\podID{#1}
  \pgfmathsetmacro{\conSize}{#2}
  \pgfmathsetmacro{\rows}{int(#3 - 1)}
  \pgfmathsetmacro{\cols}{int(#4 - 1)}
  \tikzstyle{loc} = [#5]
  \node[loc] (\podID_location) {};

  \foreach \r in {0,...,\rows}{
    \foreach \c in {0,...,\cols}{
      \pgfmathsetmacro{\vdis}{(\r * \conSize)}
      \pgfmathsetmacro{\hdis}{(\c * \conSize)}
      \node[c_full, above right= \vdis cm and \hdis cm of \podID_location] (\podID C\r \c) {};
    }
  }

  % randomly generate color for pod
  \pgfmathparse{rnd}
  \pgfmathtruncatemacro{\cc}{(\rows + \cols)*0.4}
  \xdefinecolor{rColor}{rgb}{\cc, 0.7, \pgfmathresult}

  \node[pod, fill=rColor, fit=(\podID C00)(\podID C\rows \cols)] (\podID) {};
}

\newcommand{\project}[5] {
  \def\projectID{#1}
  \pgfmathsetmacro{\projectSize}{#2}
  \pgfmathsetmacro{\numCons}{(#3)}
  \pgfmathsetmacro{\numPods}{#4}
  \pgfmathsetmacro{\podRange}{int(\numPods - 1)}
   
  %\pgfmathsetmacro{\numCons}{(#3 - 1)}
  %\pgfmathsetmacro{\numPods}{(#4 - 1)}
  \tikzstyle{prevLoc} = [#5]
  \node[#5] (\projectID_location) {};  
  
  \pgfmathsetmacro{\consPerPod}{(\numCons / \numPods)}
  %\pgfmathsetmacro{\cl}{\nnumCons }
  %\foreach \np [count=\S from 1] in {0,...,\numPods}{
  \foreach \np in {0,...,\podRange}{
    \pgfmathsetmacro{\sep}{\np * ((\consPerPod+1)*\projectSize)}
    %\pgfmathsetmacro{\cons}{int((\numCons - (\consPerPod * \np)) / \numPods) + 1 }
    \pod{\projectID P\np}{\projectSize}{1}{\consPerPod}{right= 0cm and \sep cm of \projectID_location}
    %\node[right= 0cm and \sep cm of \projectID_location] (\projectID P\np) {\cl};
  }

  \node[NS, fit=(\projectID P0)(\projectID P\podRange)] (\projectID) {};
  %\node[above = 0cm of \projectID] (\projectID_label) {Project};
}

\newcommand{\router}[3]{
  \def\routerID{#1}
  \def\size{#2}
  \tikzstyle{loc} = [#3]
  
  \pgfmathsetmacro{\height}{\size * 0.25}
  \tikzstyle{router} = [rectangle, black, rounded corners, minimum height = \height cm, minimum width = \size cm]

  \node[router, fill=black!30, loc] (in_router) {};
  \node[thick, draw, rounded corners, black, fit=(in_router)] (\routerID) {};
  \node[router, fill=black!30, loc] (in_router2) {};

  \tikzstyle{light} = [circle, green, minimum width = 0.05ex]

  \node[light] (dot1) at (\routerID.center) {};
  \node[light, left = 0cm and 0cm of dot1] (dot2) {};
  \node[light, left = 0cm and 0cm of dot2] (dot3) {};

  \tikzstyle{antenna} = [draw,rectangle, rounded corners, fill=black!20, rotate=0, minimum height = 1cm]
  \node[antenna, above right= 0cm and 0.75cm of \routerID.north] (air) {};
  \node[above= 0cm of \routerID] () {Router};
  
}

% Make 3D template stack
\newcommand{\template}[4]{
  \def\tempID{#1}
  \pgfmathsetmacro{\size}{#2}
  \def\color{#3}
  \tikzstyle{location} = [#4]

  \def\x{\size * 0.4}
  \def\y{\size * 0.5}
  \def\z{\size * 1.0}

  \tikzstyle{coor} = [shape=coordinate]
  \node[circle, location] (\tempID) {};
 
  \node[coor, below left =\x cm and \y cm of \tempID] (A_\tempID) {A};
  \node[coor, right=0.0cm and \z cm of A_\tempID] (B_\tempID) {B};
  \node[coor, above right=\y cm and \x cm of A_\tempID] (C_\tempID) {C};
  \node[coor, above right=\y cm and \x cm of B_\tempID] (D_\tempID) {D};

  \tikzstyle{face} = [black, opacity=0.5]

  \draw[face, fill=\color!30] (A_\tempID) -- (C_\tempID) -- (D_\tempID) -- (B_\tempID) -- cycle; % Bottom Face

  \node[coor] (\tempID_center) at ($(A_\tempID)!0.5!(D_\tempID)$)  {};
  \node[coor] (\tempID_north) at ($(C_\tempID)!0.5!(D_\tempID)$)  {};
  \node[coor] (\tempID_east) at ($(B_\tempID)!0.5!(D_\tempID)$)  {};
  \node[coor] (\tempID_south) at ($(A_\tempID)!0.5!(B_\tempID)$)  {};
  \node[coor] (\tempID_west) at ($(A_\tempID)!0.5!(C_\tempID)$)  {};

  \node[fit=(A_\tempID)(D_\tempID)] (\tempID_fit) {};
  %\tikzstyle{cube}= [fit=(S_\id)(A_\id)(B_\id)(C_\id)(D_\id)(E_\id)(F_\id)(G_\id)];
  %\node[circle, draw, location] (test\id) {};
  
  % Writing
  %\node[above left =  0.2cm and 0.5cm of \tempID_center] (\tempID A1) {};
  %\node[above right = 0.2cm and 0.1cm of \tempID_center] (\tempID A2) {};
  %\draw  (\tempID A1) -- (\tempID A2);
  %\node[above left = 0.0cm and  0.7cm of \tempID_center] (\tempID B1) {};
  %\node[above right = 0.0cm and 0.3cm of \tempID_center] (\tempID B2) {};
  %\draw  (\tempID B1) -- (\tempID B2);
  %\node[below left = 0.0cm and  0.9cm of \tempID_center] (\tempID C1) {};
  %\node[below right = 0.0cm and 0.25cm of \tempID_center] (\tempID C2) {};
  %\draw  (\tempID C1) -- (\tempID C2);
  %\node[below left = 0.1cm and  0.9cm of \tempID_center] (\tempID D1) {};
  %\node[below right = 0.1cm and 0.2cm of \tempID_center] (\tempID D2) {};
  %\draw  (\tempID D1) -- (\tempID D2);
}

% Make 3D template stack
\newcommand{\catalog}[4] {
  \def\catID{#1}
  \pgfmathsetmacro{\tempSize}{#2}
  \pgfmathsetmacro{\tempNum}{int(#3 - 1)}
  \tikzstyle{loc} = [#4]
  \node[loc] (\catID_location) {};
    %\template{TTT}{1.2}{blue!50}{#4};

  \foreach \i in {0,...,\tempNum}{
    \pgfmathsetmacro{\vvdis}{\i * 0.25}
    \template{\catID T\i}{\tempSize}{blue}{above = \vvdis cm of \catID_location};
  }

  \node[fit=(\catID T0_fit)(\catID T\tempNum_fit)] (\catID) {};
  \node[above = -0.25cm of \catID] (\catID_label) {\small Catalog};
}

%\newcommand*{\var}[2]{%
%    \newcommand*{#1}{}% Error if already defined
%    \pgfmathsetmacro{#1}{#2}%
%}%
\newcommand{\openshift}[4] {
  \def\id{#1}
  \def\width{#2}
  \pgfmathsetmacro{\height}{#2 / 3}
  \tikzstyle{loc} = [#3]
  \node[#3] (LOCATION) {};
  \pgfmathsetmacro{\dis}{\height}
  
  \pgfmathsetmacro{\owidth}{\width / 3} 
  \pgfmathsetmacro{\oheight}{\height / 4}
   
  \tikzstyle{mw} = [minimum width=\width cm]
  \tikzstyle{mh} = [minimum height= \height cm]
  \tikzstyle{box} = [rectangle, thick, rounded corners, text= white]
  \tikzstyle{layer} = [box, mw, mh]
  
  \tikzstyle{OS_LAYER} = [box, OS_COLOR, minimum width = \owidth cm, minimum height = \oheight cm]

  \node[USR_COLOR, layer, label={[label distance=-0.55cm]270:\color{white}Userspace}, above = 0.45 cm  of LOCATION ] (\id_USR) {};
  \node[KRN_COLOR, layer, below = 0cm of \id_USR] (\id_KRN) {Kernel};
  \node[HW_COLOR,  layer, below = 0cm of \id_KRN] (\id_HW) {Hardware};
  
  \node[OS_LAYER, above=0.5cm of \id_USR.south] (\id_label) {\small OpenShift};
  
  \node [below = 0cm of \id_HW.south] (\id_HM) {Physical Machine};
  \path[->, thin, to path={-| (\tikztotarget)}]
    (\id_HM.west) edge (\id_HW.south west)
    (\id_HM.east) edge (\id_HW.south east);
  
  %\node [rotate=90, left = of \id_KRN.west, anchor=south] (\id_HM) {Host Machine};
  %    \path[->, thin, to path={|- (\tikztotarget)}]
  %    (\id_HM.east) edge (\id_USR.north west)
  %    (\id_HM.west) edge (\id_HW.south west);
      
  \node[fit=(\id_USR)(\id_KRN)(\id_HW)] (\id) {};
}







