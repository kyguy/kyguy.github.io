%%%%%%%%%%%%%%%%%%%%%%%%%%%%%%%%%%%%%%%%%%%%%%%%%
% PREAMBLE
%%%%%%%%%%%%%%%%%%%%%%%%%%%%%%%%%%%%%%%%%%%%%%%%%
\usepackage{hyperref}
\usepackage{listings}
\usepackage{color}
\usepackage[dvipsnames]{xcolor}
\usepackage{float}
\usepackage{graphicx}
\usepackage{tikz,tikz-dependency}
\usepackage{ifthen}
\usetikzlibrary{calc}
\usetikzlibrary{automata,positioning,external,shapes,arrows,chains,matrix,scopes,backgrounds}

% The following is for tikz externalization configuration which allows
% tikZ diagrams to render correctly while being processed by htlatex 
% http://tex.stackexchange.com/questions/158551/using-htlatex-with-tikz-dependency#158921
%%%%%%%%%%%%%%%%%%%%%%%%%%%%%%%%%%%%%%%%%%%%%%%%%
\tikzset{
    tex4ht inc/.style={
        /pgf/images/include external/.code={%
            \includegraphics[]{##1.svg}%
        }

    }
}
\tikzset{
 external/system call/.add={}
      ; inkscape -z -f "\image.pdf" -l "\image.svg"
}
\makeatletter
\@ifpackageloaded{tex4ht}{
    \tikzexternalize[mode=only graphics]
}{
    \tikzexternalize
}
\makeatother

% Removes section numbers
\makeatletter
\renewcommand\thesection{}
%\renewcommand\thesubsection{\@arabic\c@section.\@arabic\c@subsection}
\renewcommand\thesubsection{}
\makeatother

% lstlisting configs for code syntax formatting
\lstdefinestyle{Java} {
  frame=tb,
  language=Java,
  aboveskip=3mm,
  belowskip=3mm,
  showstringspaces=false,
  columns=flexible,
  basicstyle={\large\ttfamily},
  numbers=none,
  numberstyle=\textcolor{gray},
  keywordstyle=\textcolor{red!75},
  commentstyle=\textcolor{dkgreen},
  stringstyle=\textcolor{blue},
  moredelim=[is][\textcolor{black!75}]{|}{|},
  breaklines=false,
  breakatwhitespace=true,
  tabsize=3
}
%%%%%%%%%%%%%%%%%%%%%%%%%%%%%%%%%%%%%%%%%%%%%%%%%
% COLOR DEFINITIONS
%%%%%%%%%%%%%%%%%%%%%%%%%%%%%%%%%%%%%%%%%%%%%%%%%
\tikzstyle{red}     = [fill=red!30!white]
\tikzstyle{orange}  = [fill=orange!30!white]
\tikzstyle{yellow}  = [fill=yellow!30!white]
\tikzstyle{green}   = [fill=green!30!white]
\tikzstyle{blue}    = [fill=blue!30!white]
\tikzstyle{teal}    = [fill=teal!30!white]
\tikzstyle{purple}  = [fill=purple!30!white]
\tikzstyle{magenta} = [fill=magenta!30!white]
\tikzstyle{gray}    = [fill=gray!30!white]
\tikzstyle{black}   = [fill=black!30!white]

%%%%%%%%%%%%%%%%%%%%%%%%%%%%%%%%%%%%%%%%%%%%%%%%%
% DEFINE STACK COLORS
%%%%%%%%%%%%%%%%%%%%%%%%%%%%%%%%%%%%%%%%%%%%%%%%%
\definecolor{DEFINE_USR_COLOR}{HTML}{006699}
\definecolor{DEFINE_KRN_COLOR}{HTML}{003366}
\definecolor{DEFINE_HW_COLOR}{HTML}{000033}
\definecolor{DEFINE_CET_COLOR}{HTML}{339999}

\tikzstyle{DK_COLOR}  = [fill=blue!30!white]
\tikzstyle{ARQ_COLOR} = [fill=gray!30!white]
\tikzstyle{OS_COLOR}  = [fill=black!50!white]
\tikzstyle{USR_COLOR} = [fill=DEFINE_USR_COLOR]
\tikzstyle{KRN_COLOR} = [fill=DEFINE_KRN_COLOR]
\tikzstyle{HW_COLOR}  = [fill=DEFINE_HW_COLOR]
\tikzstyle{CET_COLOR} = [fill=DEFINE_CET_COLOR]
\tikzstyle{HM_COLOR}  = [fill=cyan!30!white]
%%%%%%%%%%%%%%%%%%%%%%%%%%%%%%%%%%%%%%%%%%%%%%%%%
% STACK DIAGRAM DEFINITIONS
%%%%%%%%%%%%%%%%%%%%%%%%%%%%%%%%%%%%%%%%%%%%%%%%%
\tikzstyle{layer}      = [rectangle, thick, rounded corners]
\tikzstyle{core_stack} = [layer, minimum width=11cm,minimum height=1cm, text = white]
\tikzstyle{user_stack} = [layer, minimum width=4cm,minimum height=1cm]
\tikzstyle{app_stack}  = [layer, minimum width=4.5cm,minimum height=3cm]
\tikzstyle{container}  = [layer, purple,minimum width=0.5cm,minimum height=0.5cm]

\tikzstyle{OS_LAYER}  = [app_stack, mw, OS_COLOR, label={[label distance=-0.75cm]270:OpenShift}]
\tikzstyle{ARQ_LAYER} = [app_stack, ARQ_COLOR,label={[label distance=-0.75cm]270:Arquillian}]
\tikzstyle{CET_LAYER} = [user_stack, CET_COLOR]
\tikzstyle{USR_LAYER} = [core_stack, USR_COLOR, minimum height=4.5cm, label={[label distance=-0.75cm]270:\color{white}Userspace}]
\tikzstyle{KRN_LAYER} = [core_stack, KRN_COLOR, below= 0cm of USR]
\tikzstyle{HW_LAYER}  = [core_stack, HW_COLOR, below of=KRN]
\tikzstyle{DK_LAYER}  = [user_stack, DK_COLOR]
\tikzstyle{HM_LAYER}  = [user_stack, HM_COLOR]

%%% NEW DEFS
\tikzstyle{OS_LAYER2} = [app_stack, OS_COLOR,fill=black!50!white, label={[label distance=-0.75cm]270:OpenShift}]


\tikzstyle{c_temp} = [layer,minimum width=0.5cm,minimum height=0.5cm]
\tikzstyle{c_empt} = [c_temp]
\tikzstyle{c_move} = [c_temp, draw, dotted]
\tikzstyle{c_full} = [c_temp, purple, draw=black, thin]
\tikzstyle{mw}     = [minimum width=5.0cm]

%%%%%%%%%%%%%%%%%%%%%%%%%%%%%%%%%%%%%%%%%%%%%%%%
% Tikz Functions
%%%%%%%%%%%%%%%%%%%%%%%%%%%%%%%%%%%%%%%%%%%%%%%%
\newcommand{\moby}[4] {
  \xdef\cmax{7}
  \def\id{#1}                   % #1 = id 
  \pgfmathsetmacro\nc{int(#2)}  % #2 = number of containers
  \pgfmathsetmacro\nfc{int(#3)} % #3 = number of full containers 
                                % #4 = location
  % Initialize container nodes
  \foreach \x in {0,...,7}{
    \node[] (C\x_\id) {};
  }
  % Instantiate moby and container nodes
  \node[DK_LAYER, DK_COLOR, #4] (DK_\id) {Moby};
  \node[above left= 0.1cm and 0cm of DK_\id.north west] (sp) {};

  \xdef\sp{sp};     % starting position
  \xdef\ct{c_full}; % container type
  \foreach \x in {0,...,\cmax} {
    {\ifthenelse{\x < \nc}
      { {\ifthenelse{\x < \nfc }
          {\xdef\ct{c_full};}
          {\xdef\ct{c_move};}
        }
      }
      { \xdef\ct{c_empt};}
    }
    \node[\ct, right=of \sp] (C\x_\id) {};
    \xdef\sp{C\x_\id};
  };

  {\ifthenelse{\nc < 4}{}{
  %Container label with spanning arrows 
  \node[above= 0.5cm of DK_\id.north ] (C_LABEL_\id) {Containers};
  \path[->, to path={-| (\tikztotarget)}]
  (C_LABEL_\id.west) edge (C0_\id.north)
  (C_LABEL_\id.east) edge (C7_\id.north);
  }};
}
\newcommand{\host}[4]{
  \def\id{#1}  % #1 = id
  \def\nc{#2}  % #2 = number of containers
  \def\nfc{#3} % #3 = number of full containers
               % #4 = location
  % Initialize nodes
  \foreach \x in {0,...,7}
    \node[] (C\x_\id) {};
    \node[] (DK_\id) {};
    \node[] (OS_\id) {};

  \node[OS_LAYER,fit=(DK_\id)] at (#4) (OS_\id) {};
  \moby{#1}{#2}{#3}{above = 0cm of OS_\id.center, anchor=center};
  \node[HM_LAYER, mw, below= 0cm of OS_\id.south, anchor=north] (HM_\id) {Host Machine \id};
}

\newcommand{\OpenShift}[4]{
  \def\id{#1}  % #1 = id
  \def\nc{#2}  % #2 = number of containers
  \def\nfc{#3} % #3 = number of full containers
               % #4 = location
  % Initialize nodes
  \foreach \x in {0,...,7}
    \node[] (C\x_\id) {};
    \node[] (DK_\id) {};
    \node[] (OS_\id) {};

  \node[OS_LAYER,fit=(DK_\id), #4] (OS_\id) {};
  \moby{#1}{#2}{#3}{above = 0cm of OS_\id.center, anchor=center};
}

\newcommand{\Arquillian}[2]{
  \def\id{#1}  % #1 = id
               % #2 = location

  \node[] (CET_\id) {};
  \node[ARQ_LAYER, fit=(CET_\id), #2] (ARQ_\id) {};
  \node[CET_LAYER, above = 0cm of ARQ_\id.center, anchor=center] (CET_\id) {CE-Testsuite};
}

\newcommand{\pods}[5]{
  \xdef\id{#1}
  \pgfmathtruncatemacro{\numPods}{#2-1}
  \xdef\numCons{#3-1};
  \pgfmathtruncatemacro{\cpp}{2 - 1} %containers per pod
  \xdef\cfit{} % containers for pod to fit
  \xdef\pfit{} % pods for namespace to fit

  \node[#4] (BASE) {};
  \xdef\lcp{BASE}; % last container position
  \xdef\sbc{}; % space between container
  \pgfmathsetmacro\r{#3}
 % Initialize pods
  \foreach \x in {0,...,\numPods}{
    \pgfmathsetmacro\pods{int(#2 - \x))}
    {\ifthenelse{\r > \pods }
    { % if 
      \xdef\numCons{1}
      \pgfmathsetmacro\r{int(\r - 2)}
      \xdef\r{\r}
    }
    { % else
      \xdef\numCons{0};
      \pgfmathsetmacro\r{int(\r - 1)}
      \xdef\r{\r}
    }
    };
    %\node[] at (-5,\x) {\r \pods \numCons};

    \foreach \y in {0,...,\numCons}{
      \node[c_full, right= \sbc of \lcp] (P\x_C\y_#1) {};
      \xdef\lcp{P\x_C\y_#1}
      \xdef\cfit{\cfit(P\x_C\y_#1)} % add container to be fitted
      \xdef\sbc{}; %space between containers
    }

    % randomly generate color for pod
    \pgfmathparse{rnd}
    \pgfmathtruncatemacro{\cc}{(\id)*0.4}
    \xdefinecolor{rColor}{rgb}{\cc, 0.7, \pgfmathresult}

    \node[pod, fill=rColor, fit=\cfit] (P\x_\id) {};
    %\pgfmathparse{rnd * \x}

    \xdef\cfit{};
    \xdef\sbc{0cm and 0.5cm}; %space between containers
    \xdef\pfit{\pfit(P\x_\id)} % add container to be fitted
    }
    #5
}

\newcommand{\namespace}[4]{
  \pods{#1}{#2}{#3}{#4}{\node[fit=\pfit] (NS_#1) {};};
  \node[NS, fit=\pfit] (NS_#1) {};
  \node[above=0cm of NS_#1.north] (NS_#1_LABEL) {Namespace #1};
  %\node[NS, fill=gray!10, fit=\pfit] (NS_#1) {};
  \pods{#1}{#2}{#3}{#4}{};
  %\pods{#1}{#2}{#3}{#4}{};
}
