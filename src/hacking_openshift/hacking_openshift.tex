%%%%%%%%%%%%%%%%%%%%%%%%%%%%%%%%%%%%%%%%%%%%%%%%%
% hacking_openshift.tex
% 
% Latex source to be converted to HTML by tex4ht
%
%%%%%%%%%%%%%%%%%%%%%%%%%%%%%%%%%%%%%%%%%%%%%%%%%
\documentclass{article}

%%%%%%%%%%%%%%%%%%%%%%%%%%%%%%%%%%%%%%%%%%%%%%%%%
% PREAMBLE
%%%%%%%%%%%%%%%%%%%%%%%%%%%%%%%%%%%%%%%%%%%%%%%%%
% Import common configurations and custom functions
%%%%%%%%%%%%%%%%%%%%%%%%%%%%%%%%%%%%%%%%%%%%%%%%%
% PREAMBLE
%%%%%%%%%%%%%%%%%%%%%%%%%%%%%%%%%%%%%%%%%%%%%%%%%
\usepackage{hyperref}
\usepackage{listings}
\usepackage{color}
\usepackage[dvipsnames]{xcolor}
\usepackage{float}
\usepackage{graphicx}
\usepackage{tikz,tikz-dependency}
\usepackage{ifthen}
\usetikzlibrary{calc}
\usetikzlibrary{automata,positioning,external,shapes,arrows,chains,matrix,scopes,backgrounds}

% The following is for tikz externalization configuration which allows
% tikZ diagrams to render correctly while being processed by htlatex 
% http://tex.stackexchange.com/questions/158551/using-htlatex-with-tikz-dependency#158921
%%%%%%%%%%%%%%%%%%%%%%%%%%%%%%%%%%%%%%%%%%%%%%%%%
\tikzset{
    tex4ht inc/.style={
        /pgf/images/include external/.code={%
            \includegraphics[]{##1.svg}%
        }

    }
}
\tikzset{
 external/system call/.add={}
      ; inkscape -z -f "\image.pdf" -l "\image.svg"
}
\makeatletter
\@ifpackageloaded{tex4ht}{
    \tikzexternalize[mode=only graphics]
}{
    \tikzexternalize
}
\makeatother

% Removes section numbers
\makeatletter
\renewcommand\thesection{}
%\renewcommand\thesubsection{\@arabic\c@section.\@arabic\c@subsection}
\renewcommand\thesubsection{}
\makeatother

% lstlisting configs for code syntax formatting
\lstdefinestyle{Java} {
  frame=tb,
  language=Java,
  aboveskip=3mm,
  belowskip=3mm,
  showstringspaces=false,
  columns=flexible,
  basicstyle={\large\ttfamily},
  numbers=none,
  numberstyle=\textcolor{gray},
  keywordstyle=\textcolor{red!75},
  commentstyle=\textcolor{dkgreen},
  stringstyle=\textcolor{blue},
  moredelim=[is][\textcolor{black!75}]{|}{|},
  breaklines=false,
  breakatwhitespace=true,
  tabsize=3
}
%%%%%%%%%%%%%%%%%%%%%%%%%%%%%%%%%%%%%%%%%%%%%%%%%
% COLOR DEFINITIONS
%%%%%%%%%%%%%%%%%%%%%%%%%%%%%%%%%%%%%%%%%%%%%%%%%
\tikzstyle{red}     = [fill=red!30!white]
\tikzstyle{orange}  = [fill=orange!30!white]
\tikzstyle{yellow}  = [fill=yellow!30!white]
\tikzstyle{green}   = [fill=green!30!white]
\tikzstyle{blue}    = [fill=blue!30!white]
\tikzstyle{teal}    = [fill=teal!30!white]
\tikzstyle{purple}  = [fill=purple!30!white]
\tikzstyle{magenta} = [fill=magenta!30!white]
\tikzstyle{gray}    = [fill=gray!30!white]
\tikzstyle{black}   = [fill=black!30!white]

%%%%%%%%%%%%%%%%%%%%%%%%%%%%%%%%%%%%%%%%%%%%%%%%%
% DEFINE STACK COLORS
%%%%%%%%%%%%%%%%%%%%%%%%%%%%%%%%%%%%%%%%%%%%%%%%%
\definecolor{DEFINE_USR_COLOR}{HTML}{006699}
\definecolor{DEFINE_KRN_COLOR}{HTML}{003366}
\definecolor{DEFINE_HW_COLOR}{HTML}{000033}
\definecolor{DEFINE_CET_COLOR}{HTML}{339999}

\tikzstyle{DK_COLOR}  = [fill=blue!30!white]
\tikzstyle{ARQ_COLOR} = [fill=gray!30!white]
\tikzstyle{OS_COLOR}  = [fill=black!50!white]
\tikzstyle{USR_COLOR} = [fill=DEFINE_USR_COLOR]
\tikzstyle{KRN_COLOR} = [fill=DEFINE_KRN_COLOR]
\tikzstyle{HW_COLOR}  = [fill=DEFINE_HW_COLOR]
\tikzstyle{CET_COLOR} = [fill=DEFINE_CET_COLOR]
\tikzstyle{HM_COLOR}  = [fill=cyan!30!white]
%%%%%%%%%%%%%%%%%%%%%%%%%%%%%%%%%%%%%%%%%%%%%%%%%
% STACK DIAGRAM DEFINITIONS
%%%%%%%%%%%%%%%%%%%%%%%%%%%%%%%%%%%%%%%%%%%%%%%%%
\tikzstyle{layer}      = [rectangle, thick, rounded corners]
\tikzstyle{core_stack} = [layer, minimum width=11cm,minimum height=1cm, text = white]
\tikzstyle{user_stack} = [layer, minimum width=4cm,minimum height=1cm]
\tikzstyle{app_stack}  = [layer, minimum width=4.5cm,minimum height=3cm]
\tikzstyle{container}  = [layer, purple,minimum width=0.5cm,minimum height=0.5cm]

\tikzstyle{OS_LAYER}  = [app_stack, mw, OS_COLOR, label={[label distance=-0.75cm]270:OpenShift}]
\tikzstyle{ARQ_LAYER} = [app_stack, ARQ_COLOR,label={[label distance=-0.75cm]270:Arquillian}]
\tikzstyle{CET_LAYER} = [user_stack, CET_COLOR]
\tikzstyle{USR_LAYER} = [core_stack, USR_COLOR, minimum height=4.5cm, label={[label distance=-0.75cm]270:\color{white}Userspace}]
\tikzstyle{KRN_LAYER} = [core_stack, KRN_COLOR, below= 0cm of USR]
\tikzstyle{HW_LAYER}  = [core_stack, HW_COLOR, below of=KRN]
\tikzstyle{DK_LAYER}  = [user_stack, DK_COLOR]
\tikzstyle{HM_LAYER}  = [user_stack, HM_COLOR]

%%% NEW DEFS
\tikzstyle{OS_LAYER2} = [app_stack, OS_COLOR,fill=black!50!white, label={[label distance=-0.75cm]270:OpenShift}]


\tikzstyle{c_temp} = [layer,minimum width=0.5cm,minimum height=0.5cm]
\tikzstyle{c_empt} = [c_temp]
\tikzstyle{c_move} = [c_temp, draw, dotted]
\tikzstyle{c_full} = [c_temp, purple, draw=black, thin]
\tikzstyle{mw}     = [minimum width=5.0cm]

%%%%%%%%%%%%%%%%%%%%%%%%%%%%%%%%%%%%%%%%%%%%%%%%
% Tikz Functions
%%%%%%%%%%%%%%%%%%%%%%%%%%%%%%%%%%%%%%%%%%%%%%%%
\newcommand{\moby}[4] {
  \xdef\cmax{7}
  \def\id{#1}                   % #1 = id 
  \pgfmathsetmacro\nc{int(#2)}  % #2 = number of containers
  \pgfmathsetmacro\nfc{int(#3)} % #3 = number of full containers 
                                % #4 = location
  % Initialize container nodes
  \foreach \x in {0,...,7}{
    \node[] (C\x_\id) {};
  }
  % Instantiate moby and container nodes
  \node[DK_LAYER, DK_COLOR, #4] (DK_\id) {Moby};
  \node[above left= 0.1cm and 0cm of DK_\id.north west] (sp) {};

  \xdef\sp{sp};     % starting position
  \xdef\ct{c_full}; % container type
  \foreach \x in {0,...,\cmax} {
    {\ifthenelse{\x < \nc}
      { {\ifthenelse{\x < \nfc }
          {\xdef\ct{c_full};}
          {\xdef\ct{c_move};}
        }
      }
      { \xdef\ct{c_empt};}
    }
    \node[\ct, right=of \sp] (C\x_\id) {};
    \xdef\sp{C\x_\id};
  };

  {\ifthenelse{\nc < 4}{}{
  %Container label with spanning arrows 
  \node[above= 0.5cm of DK_\id.north ] (C_LABEL_\id) {Containers};
  \path[->, to path={-| (\tikztotarget)}]
  (C_LABEL_\id.west) edge (C0_\id.north)
  (C_LABEL_\id.east) edge (C7_\id.north);
  }};
}
\newcommand{\host}[4]{
  \def\id{#1}  % #1 = id
  \def\nc{#2}  % #2 = number of containers
  \def\nfc{#3} % #3 = number of full containers
               % #4 = location
  % Initialize nodes
  \foreach \x in {0,...,7}
    \node[] (C\x_\id) {};
    \node[] (DK_\id) {};
    \node[] (OS_\id) {};

  \node[OS_LAYER,fit=(DK_\id)] at (#4) (OS_\id) {};
  \moby{#1}{#2}{#3}{above = 0cm of OS_\id.center, anchor=center};
  \node[HM_LAYER, mw, below= 0cm of OS_\id.south, anchor=north] (HM_\id) {Host Machine \id};
}

\newcommand{\OpenShift}[4]{
  \def\id{#1}  % #1 = id
  \def\nc{#2}  % #2 = number of containers
  \def\nfc{#3} % #3 = number of full containers
               % #4 = location
  % Initialize nodes
  \foreach \x in {0,...,7}
    \node[] (C\x_\id) {};
    \node[] (DK_\id) {};
    \node[] (OS_\id) {};

  \node[OS_LAYER,fit=(DK_\id), #4] (OS_\id) {};
  \moby{#1}{#2}{#3}{above = 0cm of OS_\id.center, anchor=center};
}

\newcommand{\Arquillian}[2]{
  \def\id{#1}  % #1 = id
               % #2 = location

  \node[] (CET_\id) {};
  \node[ARQ_LAYER, fit=(CET_\id), #2] (ARQ_\id) {};
  \node[CET_LAYER, above = 0cm of ARQ_\id.center, anchor=center] (CET_\id) {CE-Testsuite};
}

\newcommand{\pods}[5]{
  \xdef\id{#1}
  \pgfmathtruncatemacro{\numPods}{#2-1}
  \xdef\numCons{#3-1};
  \pgfmathtruncatemacro{\cpp}{2 - 1} %containers per pod
  \xdef\cfit{} % containers for pod to fit
  \xdef\pfit{} % pods for namespace to fit

  \node[#4] (BASE) {};
  \xdef\lcp{BASE}; % last container position
  \xdef\sbc{}; % space between container
  \pgfmathsetmacro\r{#3}
 % Initialize pods
  \foreach \x in {0,...,\numPods}{
    \pgfmathsetmacro\pods{int(#2 - \x))}
    {\ifthenelse{\r > \pods }
    { % if 
      \xdef\numCons{1}
      \pgfmathsetmacro\r{int(\r - 2)}
      \xdef\r{\r}
    }
    { % else
      \xdef\numCons{0};
      \pgfmathsetmacro\r{int(\r - 1)}
      \xdef\r{\r}
    }
    };
    %\node[] at (-5,\x) {\r \pods \numCons};

    \foreach \y in {0,...,\numCons}{
      \node[c_full, right= \sbc of \lcp] (P\x_C\y_#1) {};
      \xdef\lcp{P\x_C\y_#1}
      \xdef\cfit{\cfit(P\x_C\y_#1)} % add container to be fitted
      \xdef\sbc{}; %space between containers
    }

    % randomly generate color for pod
    \pgfmathparse{rnd}
    \pgfmathtruncatemacro{\cc}{(\id)*0.4}
    \xdefinecolor{rColor}{rgb}{\cc, 0.7, \pgfmathresult}

    \node[pod, fill=rColor, fit=\cfit] (P\x_\id) {};
    %\pgfmathparse{rnd * \x}

    \xdef\cfit{};
    \xdef\sbc{0cm and 0.5cm}; %space between containers
    \xdef\pfit{\pfit(P\x_\id)} % add container to be fitted
    }
    #5
}

\newcommand{\namespace}[4]{
  \pods{#1}{#2}{#3}{#4}{\node[fit=\pfit] (NS_#1) {};};
  \node[NS, fit=\pfit] (NS_#1) {};
  \node[above=0cm of NS_#1.north] (NS_#1_LABEL) {Namespace #1};
  %\node[NS, fill=gray!10, fit=\pfit] (NS_#1) {};
  \pods{#1}{#2}{#3}{#4}{};
  %\pods{#1}{#2}{#3}{#4}{};
}


%%%%%%%%%%%%%%%%%%%%%%%%%%%%%%%%%%%%%%%%%%%%%%%%%
% DOCUMENT
%%%%%%%%%%%%%%%%%%%%%%%%%%%%%%%%%%%%%%%%%%%%%%%%%
\iffalse
#########################################################
# Jenkins
#########################################################
Need to pull rhel jenkins image
upload it to registry
curl maven
curl -O -J -L  mirror.olnevhost.net/pub/apache/maven/maven-3/3.5.2/binaries/apache-maven-3.5.2-bin.tar.gz

add it to path
vi ~/.bashrc
export PATH=$PATH:<>
##########################################################
##########################################################
Make Persistent Volumes
##########################################################
# Follow https://github.com/openshift/origin/blob/master/examples/wordpress/nfs/README.md

# INTO pv-1.yaml
apiVersion: v1
kind: PersistentVolume
metadata:
  name: pv0001
spec:
  capacity:
    storage: 1Gi
  accessModes:
    - ReadWriteOnce
    - ReadWriteMany
  persistentVolumeReclaimPolicy: Recycle
  nfs:
    server: localhost
    path: /home/data/pv0001


sudo setenforce 0
** Need to find proper ip table
##########################################################
###########################################################
oc get svc -n default | grep registry
fd

### hawkular metrics ##############################################
oc delete all --selector="metrics-infra"
oc delete templates --selector="metrics-infra"
oc delete secrets --selector="metrics-infra"
oc delete pvc --selector="metrics-infra"
oc delete sa --selector="metrics-infra"

# Change version to v5
https://raw.githubusercontent.com/openshift/openshift-ansible/master/roles/openshift_hosted_templates/files/v1.4/enterprise/metrics-deployer.yaml

oc create -f metrics-deployer-setup.yaml -n openshift-infra

oc adm policy add-role-to-user cluster-admin system:serviceaccount:openshift-infra:metrics-deployer
oc adm policy add-role-to-user edit system:serviceaccount:openshift-infra:metrics-deployer
oc adm policy add-cluster-role-to-user cluster-admin system:serviceaccount:openshift-infra:heapster
oc adm policy add-cluster-role-to-user cluster-reader system:serviceaccount:openshift-infra:heapster
oc adm policy add-role-to-user view system:serviceaccount:openshift-infra:hawkular -n openshift-infra
oc secrets new metrics-deployer nothing=/dev/null

oc process -f metrics-deployer.yaml -p HAWKULAR_METRICS_HOSTNAME=metrics.127.0.0.1.xip.io  -p USE_PERSISTENT_STORAGE=false | oc create -f -

oc process -f metrics-deployer.yaml -p HAWKULAR_METRICS_HOSTNAME=metrics-openshift-infra.192.168.0.20.xip.io  -p USE_PERSISTENT_STORAGE=false | oc create -f -

oc process -f metrics-deployer.yaml -p HAWKULAR_METRICS_HOSTNAME=hawkular-metrics.example.com -p USE_PERSISTENT_STORAGE=false | oc create -f -

# Hawkular agent
oc create -f deploy/openshift/hawkular-openshift-agent-configmap.yaml -n default
oc process -f deploy/openshift/hawkular-openshift-agent.yaml | oc create -n default -f -
oc adm policy add-cluster-role-to-user hawkular-openshift-agent system:serviceaccount:default:hawkular-openshift-agent

# Remove Hawkular agent
oc delete all,secrets,sa,templates,configmaps,daemonsets,clusterroles --selector=metrics-infra=agent -n default
oc delete clusterroles hawkular-openshift-agent
\fi
%%%%%%%%%%%%%%%%%%%%%%%%%%%%%%%%%%%%%%%%%%%%%%%%%
% DOCUMENT
%%%%%%%%%%%%%%%%%%%%%%%%%%%%%%%%%%%%%%%%%%%%%%%%%
\begin{document}
\centerline{\sc \large Hacking OpenShift}
\centerline{\sc Practicum for the Layperson }
\centerline{\url{https://github.com/openshift/origin}}

\section{Overview}
\hspace{3pc} Although OpenShift is very promising as a distributed software platform, it is still difficult to install and 
configure. This guide will walk through how to install OpenShift Origin as a single node cluster by source.

\begin{figure}
\begin{tikzpicture}
  \openshift{OS0}{3.2}{}{};
  \openshift{OS1}{3.2}{right = 2cm of OS0}{};
  \openshift{OS2}{3.2}{right = 2cm of OS1}{};
\end{tikzpicture}
\end{figure}

Keep in mind,
\begin{enumerate}[(1)]
  \item Installing OpenShift Origin via source
  \item Target OS: Fedora 25
\end{enumerate}

\section{Configuring Single Node}

Install Dependencies
\begin{lstlisting}[style=shell]
sudo dnf install golang golang-race make gcc zip mercurial krb5-devel bsdtar bc rsync bind-utils file jq tito createrepo openssl gpgme gpgme-devel libassuan libassuan-devel git docker -y
\end{lstlisting}

Prepare Docker
%\begin{lstlisting}[language=bash,caption={bash}]
\begin{lstlisting}[style=shell]
# PREP DOCKER
sudo vi /etc/sysconfig/docker

# Need to add correct registry ips
# ---- add lines ---
ADD_REGISTRY='--add-registry 172.30.0.0/16'
INSECURE_REGISTRY='--insecure-registry 172.30.0.0/16'
# ------------------

sudo groupadd docker
sudo usermod -aG docker $USER

# Reload group assignments
exec su -l $USER

# Start docker daemon
systemctl start docker
\end{lstlisting}

Install Golang
\begin{lstlisting}[style=shell]

# Latest golang release from https://golang.org/dl/
curl -sL  https://redirector.gvt1.com/edgedl/go/go1.9.2.linux-amd64.tar.gz | tar -C /usr/local -xz
sudo mv go /usr/local/

mkdir $HOME/go

# vi ~/.bashrc
#------ Append to following ------------
export GOPATH=$HOME/go
export OS_OUTPUT_GOPATH=1
export GOROOT=/usr/local/go

# OpenShift tools
export OS_TOOLS=$GOPATH/src/github.com/openshift/origin/_output/local/bin/linux/amd64/

export PATH=$PATH:$GOPATH/bin:$GOROOT/bin:$OS_TOOLS
#---------------------------------------

source ~/.bashrc
\end{lstlisting}

Compile OpenShift Origin source
\begin{lstlisting}[style=shell]
mkdir -p $GOPATH/src/github.com/openshift
cd $GOPATH/src/github.com/openshift

git clone https://github.com/openshift/origin.git
cd origin
git checkout tags/v3.7.0-alpha.1

# Compile OpenShift Server
make build  WHAT=cmd/openshift GOFLAGS= -ldflags "-w"

# Compile OpenShift Client
make build  WHAT=cmd/oc GOFLAGS= -ldflags "-w"
\end{lstlisting}

Expose Firewall on Host Machine

\begin{lstlisting}[style=shell]
# Expose ports necessary for OpenShift
firewall-cmd --permanent --new-zone dockerc
firewall-cmd --permanent --zone dockerc --add-source 172.17.0.0/16
firewall-cmd --permanent --zone dockerc --add-port 8443/tcp
firewall-cmd --permanent --zone dockerc --add-port 53/udp
firewall-cmd --permanent --zone dockerc --add-port 8053/udp
firewall-cmd --reload
\end{lstlisting}

\section{Starting Single Node Cluster}
Remember, this configuration is for a basic single node OpenShift instance 

\begin{lstlisting}[style=shell]
sudo ./openshift start
\end{lstlisting}

will give:

\begin{figure}
\begin{tikzpicture}[node distance = 1cm and 0cm]
  % OpenShift layer
  \tikzstyle{OS_LAYER} = [app_stack, OS_COLOR, minimum width = 14cm, minimum height = 6.5cm]
  \tikzstyle{image} = [c_full, fill=white!20, opacity=0.5]
  \tikzstyle{mw} = [minimum width=15.0cm]
  \tikzstyle{mh} = [minimum height=8.0cm]

  \node[USR_LAYER, mw, mh] (USR) {};
  \node[KRN_LAYER, mw, below = 0cm and 0cm of USR.south] (KRN) {Kernel};
  \node[HW_LAYER,  mw, below = 0cm and 0cm of KRN.north, anchor=north] (HW) {Hardware};

  \node[OS_LAYER, above=1cm of USR.south] (OS) {\Large \color{white} OpenShift};

  \tikzstyle{project} = [draw, rectangle, minimum height = 1.70cm, minimum width = 2.25cm, dotted, thick ];
  \xdef\dis{0.2};

  \node[project, below left= 0.75cm and 4.00cm of OS.center] (PRJ0) {default};
  %\node[opacity=0.8, above = -0.2cm of PRJ0.center] (PRJ0_LABEL) {default};
  \node[project, right = 0cm and \dis cm  of PRJ0] (PRJ1) {kube-public};
  \node[project, right = 0cm and \dis cm  of PRJ1] (PRJ2) {kube-system};
  \node[project, right = 0cm and \dis cm  of PRJ2] (PRJ3) {openshift};
  \node[project, right = 0cm and \dis cm  of PRJ3] (PRJ4) {openshift-infra};
  
  \node [below = 0cm of PRJ2] (PRJ_LABEL) {Projects};
  \path[link, <-, to path={|- (\tikztotarget)}]
    (PRJ0.south west) edge (PRJ_LABEL.west)
    (PRJ4.south east) edge (PRJ_LABEL.east);

\end{tikzpicture}
\end{figure}


Important to Configure OpenShift before doing anything else
\begin{lstlisting}[style=shell]
# Compiled files are kept in the following directory 
cd _output/local/bin/linux/amd64

sudo chmod 777 openshift.local.config/master/admin.kubeconfig

./oc adm registry -n default --config=openshift.local.config/master/admin.kubeconfig --service-account=registry

./oc adm policy add-scc-to-user hostnetwork -z router --config=openshift.local.config/master/admin.kubeconfig 

./oc adm router --replicas=1 --config=openshift.local.config/master/admin.kubeconfig --service-account=router

./oc adm policy add-cluster-role-to-user cluster-admin admin --config=openshift.local.config/master/admin.kubeconfig
\end{lstlisting}

\begin{figure}
\begin{tikzpicture}[node distance = 1cm and 0cm]
  % OpenShift layer
  \tikzstyle{OS_LAYER} = [app_stack, OS_COLOR, minimum width = 14cm, minimum height = 6.5cm]
  \tikzstyle{image} = [c_full, fill=white!20, opacity=0.5]
  \tikzstyle{mw} = [minimum width=15.0cm]
  \tikzstyle{mh} = [minimum height=8.0cm]

  \node[USR_LAYER, mw, mh] (USR) {};
  \node[KRN_LAYER, mw, below = 0cm and 0cm of USR.south] (KRN) {Kernel};
  \node[HW_LAYER,  mw, below = 0cm and 0cm of KRN.north, anchor=north] (HW) {Hardware};

  \node[OS_LAYER, above=1cm of USR.south] (OS) {\Large \color{white} OpenShift};


  \tikzstyle{project} = [draw, rectangle, minimum height = 1.70cm, minimum width = 2.25cm, dotted, thick ];
  \xdef\dis{0.2};

  \project{PRJ0}{0.5}{4}{2}{below left= 2cm and 6.5cm of OS.center}
  %\node[opacity=0.8, above = -0.2cm of PRJ0.center] (PRJ0_LABEL) {default};
  \node[project, right = 0cm and \dis cm  of PRJ0] (PRJ1) {kube-public};
  \node[project, right = 0cm and \dis cm  of PRJ1] (PRJ2) {kube-system};
  \node[project, right = 0cm and \dis cm  of PRJ2] (PRJ3) {openshift};
  \node[project, right = 0cm and \dis cm  of PRJ3] (PRJ4) {openshift-infra};
  
  \registry{R0}{0.5}{3}{4}{above left = 1.75cm and 0.0cm of PRJ0}  
  \router{ROUTER}{1.75}{above= 2.0cm of PRJ1}
 
  \node[below = 1cm of R0] (CONN_REG) {};
  \node[below = 0.5 cm of ROUTER] (CONN_ROU) {};

 \path[link, to path={|- (\tikztotarget)}]
    (PRJ0P0.north)    edge (CONN_REG.center)
    (CONN_REG.center) edge (R0.south)
    (PRJ0P1.north)    edge (CONN_ROU.center)
    (CONN_ROU.center) edge (ROUTER.south);
    
 \node [below = 0cm of PRJ2] (PRJ_LABEL) {Projects};
  \path[link, <-, to path={|- (\tikztotarget)}]
    (PRJ0.south west) edge (PRJ_LABEL.west)
    (PRJ4.south east) edge (PRJ_LABEL.east);

  %(C1_0.north) edge (CONN1_0.center)

  %\path[->] (PRJ0P0.north) edge (R0.255);


  %\project{PRJ0}{0.45}{4}{2}{above left= 2cm and 3cm of OS.center}
  %\node[project, above left= 2cm and 3cm of OS.center] (PRJ0) {default};
  %\node[project, below = \dis cm  of PRJ0] (PRJ1) {kube-public};
  %\node[project, below = \dis cm  of PRJ1] (PRJ2) {kube-system};
  %\node[project, below = \dis cm  of PRJ2] (PRJ3) {openshift};
  %\node[project, below = \dis cm  of PRJ3] (PRJ4) {openshift-infra};
\end{tikzpicture}
\end{figure}

 
\begin{lstlisting}[style=shell]
# Login as user: admin password: admin
oc login https://127.0.0.1:8443
\end{lstlisting}


\section{Optional Additions to Cluster}

\subsection{Adding local Persistent Volumes}

Setting up Network File System
\begin{lstlisting}
# Credit: https://github.com/openshift/origin/blob/master/examples/wordpress/nfs/README.md
# the directories in this example can grow unbounded
# use disk partitions of specific sizes to enforce storage quotas
mkdir -p /home/data/pv0001
mkdir -p /home/data/pv0002

# security needs to be permissive currently, but the export will soon be restricted 
# to the same UID/GID that wrote the data
chmod -R 777 /home/data/

# Add to /etc/exports
/home/data/pv0001 *(rw,sync)
/home/data/pv0002 *(rw,sync)

# Enable the new exports without bouncing the NFS service
exportfs -a

# -P makes the bool persistent between reboots.
$ setsebool -P virt_use_nfs 1
\end{lstlisting}

Create Persistent volume
\begin{lstlisting}[style=shell]
vi pv-1.yaml
\end{lstlisting}

\begin{lstlisting}[style=shell]
apiVersion: v1
kind: PersistentVolume
metadata:
  name: pv0001
spec:
  capacity:
    storage: 1Gi
  accessModes:
    - ReadWriteOnce
    - ReadWriteMany
  persistentVolumeReclaimPolicy: Recycle
  nfs:
    server: localhost
    path: /home/data/pv0001
\end{lstlisting}

\begin{lstlisting}[style=shell]
oc create -f pv-1.yaml
\end{lstlisting}


%\begin{lstlisting}[style=shell]
%[k@localhost ~]$ sudo oc cluster up --skip-registry-check=true
%   OpenShift server started.
%   The server is accessible via web console at:
%       https://192.168.122.120:8443
%
%   You are logged in as:
%       User:     developer
%       Password: developer
%
%   To login as administrator:
%       oc login -u system:admin
%[k@localhost ~]$
%\end{lstlisting}


%\begin{enumerate}
%  \item Router
%  \item Registry
%  \item Images
%   \item Templates%
%\end{enumerate}

\end{document}
